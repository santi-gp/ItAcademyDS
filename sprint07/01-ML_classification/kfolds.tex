\documentclass[border=2mm]{standalone}
\usepackage[latin1]{inputenc}
\usepackage{tikz}
\usepackage{pgfplots}
\pgfplotsset{width=10cm,compat=1.9}
\begin{document}
\begin{tikzpicture}[scale=1]

\draw [red,thin,|-|] (0,5) -- (12,5) node [pos=.5, below] {N�mero total de datos};

\foreach \x in {0,2.5,...,10}{
	 \foreach \y in {0,0.75,...,3}{
		\fill[orange] (\x,\y) -- (\x+2,\y) -- (\x+2,\y+0.5) --(\x,\y+0.5)  -- cycle;
	}
}
 \fill [blue] (10,3) rectangle (12,3.5);
\fill [blue] (7.5,2.25) rectangle (9.5,2.75);
\fill [blue] (5,1.5) rectangle (7,2);
\fill [blue] (2.5,.75) rectangle (4.5,1.25);
\fill [blue] (0,0) rectangle (2,.5);

\node (a) at (-1.5,0.25) {iteraci�n 1};
\node (a) at (-1.5,1) {iteraci�n 2};
\node (a) at (-1.5,1.75) {iteraci�n 3};
\node (a) at (-1.5,2.5) {iteraci�n 4};
\node (a) at (-1.5,3.25) {iteraci�n 5};

 \fill [blue] (0,-1) rectangle (2,-.5);
\node[right] (a) at (2.5,-.75) {Validaci�n};
\fill [orange] (0,-2) rectangle (2,-1.5);
\node[right] (a) at (2.5,-1.75) {Entrenamiento};
\end{tikzpicture}
\end{document}