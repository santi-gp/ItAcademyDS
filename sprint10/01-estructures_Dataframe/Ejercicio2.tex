\documentclass[a4paper,12pt]{article}
\usepackage[latin1]{inputenc}
%\usepackage[spanish]{babel}
%\usepackage{lmodern}
\usepackage{enumitem}
\usepackage{grffile}
\usepackage{makeidx,graphics,graphicx,array}
\usepackage{amsmath,amsthm,amsfonts,amssymb}
\usepackage[left=2cm,right=2cm,top=2cm,bottom=2cm]{geometry}
%\newtheorem{defi}{Definici�n}%[chapter]
%\newtheorem{col}{Corolario}[chapter]
%\newtheorem{teo}{Teorema}%[chapter]
%\newtheorem{lem}{Lema}[chapter]
%\newtheorem{prob}[teo]{Problema}%[chapter]
%\newtheorem{obs}{Observaci�n}[chapter]
%\newtheorem{eje}{Ejemplo}[chapter]
%\newtheorem{ejer}{Ejercicio}[chapter]
%\newtheorem{expe}{Experimento}[chapter]
%\renewcommand*{\proofname}{\textbf{Soluci�n}}
\title{Ejercicio 2}
\author{}
\date{}
\begin{document}
\maketitle
\section{Estad�sticas de los juegos ol�mpicos de Rio 2016}
Medallero de los juegos ol�mpicos de 2016

\section{Acerca del conjunto de datos}

\subsection{Contexto}

Este conjunto de datos contiene estad�sticas de las medallas logradas por los 11 mejores pa�ses que participaron en los juegos ol�mpicos de R�o de Janeiro 2016.

\begin{table}[!ht]
\centering
\begin{tabular}{lllllll}
\hline
index & ranking &                country & gold\_medal & silver\_medal & bronze\_medal & total \\
\hline
0  &       1 &   Estados Unidos (USA) &         46 &           37 &           38 &   121 \\
1  &       2 &      Reino Unido (GBR) &         27 &           23 &           17 &    67 \\
2  &       3 &            China (CHN) &         26 &           18 &           26 &    70 \\
3  &       4 &            Rusia (RUS) &         19 &           18 &           19 &    56 \\
4  &       5 &         Alemania (GER) &         17 &           10 &           15 &    42 \\
5  &       6 &            Jap�n (JPN) &         12 &            8 &           21 &    41 \\
6  &       7 &          Francia (FRA) &         10 &           18 &           14 &    42 \\
7  &       8 &    Corea del Sur (KOR) &          9 &            3 &            9 &    21 \\
8  &       9 &           Italia (ITA) &          8 &           12 &            8 &    28 \\
9  &      10 &        Australia (AUS) &          8 &           11 &           10 &    29 \\
10 &      13 &           Brasil (BRA) &          7 &            6 &            6 &    19 \\
\hline
\end{tabular}
\caption{}
\label{tab1}
\end{table}

\section{Contenido}
11 filas y 6 columnas.

La descripci�n de las columnas se enumera a continuaci�n.

\begin{table}[!ht]
\centering
\begin{tabular}{|c|c|}
\hline
\textbf{Variable} & \textbf{Descripci�n}\\
\hline
ranking & posici�n\\
country & pa�s\\
gold\_medal & medalla de oro\\
silver\_medal & medalla de plata\\
bronze\_medal & medalla de bronce\\
total & total\\
\hline
\end{tabular}
\caption{}
\label{tab2}
\end{table}
\subsection{Informaci�n del DataFrame}
De la tabla \ref{tab3} podemos observar que nuestras variables son \textit{strings}, pero si observamos la tabla \ref{tab1} a excepci�n de la columna \textbf{country} las dem�s parecen ser num�ricas.

\begin{table}[!ht]
\centering
\begin{tabular}{llll}
\#  & Column  &       Non-Null Count &   Dtype\\ 
--- &  ------   &      --------------  ----- \\ 
 0  &  ranking    &    11 non-null   &   object \\
 1  &  country     &   11 non-null   &   object\\
 2  &   gold\_medal  &   11 non-null  &    object\\
 3  &  silver\_medal &  11 non-null   &   object\\
 4  &  bronze\_medal &  11 non-null    &  object\\
 5  &  total       &   11 non-null    &  object\\

\end{tabular}
\caption{}
\label{tab3}
\end{table}
\section{Conclusi�n}
Hemos visto la manera de obtener datos a trav�s de WebScraping, pero para hacer estudios como un modelado estos datos no est�n listos para su estudio, por ejemplo, en los datos de la tabla \ref{tab1} para hacer el preprocesado se tienen que transformar estas variables, \textbf{country} como categ�rica y el resto como num�ricas.
\end{document}
